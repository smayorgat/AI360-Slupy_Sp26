\documentclass[10pt]{article}
%\documentclass[10pt]{scrartcl}

\usepackage[marginparwidth=35pt,textheight=23cm,voffset=-1.2cm,footskip=1.3cm,voffset=0.2cm]{geometry}


\usepackage{amsmath, amssymb, amsthm}

\usepackage{mathrsfs}

%\usepackage{mathabx}

\usepackage[shortlabels]{enumitem}

%\usepackage[dvips]{graphicx}
%\usepackage[dvipsnames]{xcolor}

%\usepackage{commath}

%\usepackage{relsize}

%\usepackage[cmtip,all]{xy}

\usepackage[T2A]{fontenc}
\usepackage[utf8x]{inputenc}
\usepackage[russian]{babel}

\usepackage[activate={true,nocompatibility},final,tracking=true,kerning=true,factor=1100,stretch=10,shrink=10]{microtype}


 
%Hyphenation rules
%--------------------------------------
\usepackage{hyphenat}

%\usepackage{cm-super}

%\hyphenation{ма-те-ма-ти-ка вос-ста-нав-ли-вать}

%\setlength\parindent{0pt}

\usepackage{tikz}
\usetikzlibrary{calc,decorations.pathreplacing,arrows.meta,patterns,shapes.geometric}

\DeclareMathOperator*{\argmax}{arg\,max}
\DeclareMathOperator*{\argmin}{arg\,min}

\newcommand{\bfit}[1]{\textbf{\textit{#1}}}

\newtheorem{thm}{Теорема}[section]

\newtheorem{defn}[thm]{Определение}
\newtheorem{prop}[thm]{Предложение}
\newtheorem{lemma}[thm]{Лемма}
\newtheorem{cor}[thm]{Следствие}

\newtheoremstyle{boldnormal}
    {\dimexpr\topsep/2\relax} % space above
    {\dimexpr\topsep/2\relax} % space below
    {\normalfont}  % body font (normal, not italic)
    {}             % indent amount
    {\bfseries}    % theorem head font
    {.}            % punctuation after theorem head
    {.5em}         % space after theorem head
    {}


\theoremstyle{boldnormal}
\newtheorem{zadachsemm}{Семинарная задача}[section] 
\newtheorem{zadachkontt}{Контрольная задача}[section] 

\newcommand{\N}{{\ensuremath{\mathbb{N}}}}
\newcommand{\Z}{{\ensuremath{\mathbb{Z}}}}
\newcommand{\Q}{{\ensuremath{\mathbb{Q}}}}
\newcommand{\I}{{\ensuremath{\mathbb{I}}}}
\newcommand{\R}{{\ensuremath{\mathbb{R}}}}
%\newcommand{\C}{{\ensuremath{\mathbb{C}}}}

\newcommand{\PP}{{\ensuremath{\mathbb{P}}}}


\newcommand{\EE}{{\ensuremath{\mathbb{E}}}}

\newcommand{\eps}{\varepsilon}

\newcommand{\dtorus}{{\ensuremath{\mathbb{T}^d}}}
\newcommand{\wasstwospace}{{\ensuremath{\mathscr{P}(\mathbb{T}^d)}}}

\usepackage[backend=biber,style=alphabetic]{biblatex}

\newtheoremstyle{boldremark}
    {\dimexpr\topsep/2\relax} % space above
    {\dimexpr\topsep/2\relax} % space below
  {}          % body font
    {}          % indent amount
    {\bfseries} % theorem head font
    {.}         % punctuation after theorem head
    {.5em}      % space after theorem head
    {}  

\theoremstyle{boldremark}
\newtheorem{rem}[thm]{Примечание}
\newenvironment{remark}
  {\pushQED{\qed}\renewcommand{\qedsymbol}{$\diamond$}\rem}
  {\popQED\endrem}

\newtheorem{hypo}[thm]{Условие}
\newenvironment{hypp}
  {\pushQED{\qed}\renewcommand{\qedsymbol}{$\lhd$}\hypo}
  {\popQED\endhypo}

\newtheorem{examplex}[thm]{Пример}
\newenvironment{example}
  {\pushQED{\qed}\renewcommand{\qedsymbol}{$\triangle$}\examplex}
  {\popQED\endexamplex}


\newenvironment{zadachsem}
  {\pushQED{\qed}\renewcommand{\qedsymbol}{$\triangleleft$}\zadachsemm}
  {\popQED\endzadachsemm}

\newenvironment{zadachkont}
  {\pushQED{\qed}\renewcommand{\qedsymbol}{$\triangleright$}\zadachkontt}
  {\popQED\endzadachkontt}



 
\IfFileExists{C:/Users/s.mayorga/YandexDisk/Documents/mybibfile.bib}{%
    \addbibresource{C:/Users/s.mayorga/YandexDisk/Documents/mybibfile.bib}%
}{%
    \IfFileExists{/Users/sergio/Yandex.Disk.localized/Documents/mybibfile.bib}{%
        \addbibresource{/Users/sergio/Yandex.Disk.localized/Documents/mybibfile.bib}%
    }{%
        \typeout{Neither full path exists, trying local mybibfile.bib}%
        \addbibresource{mybibfile.bib}%
    }%
} 
 
\usepackage{doi}

\usepackage{hyperref}


\author{Cергей МАЙОРГА ТАТАРИН}
\date{\today}
\title{Лекции 1 и 2 \\ Теория случайных процессов \\ Программа AI360, Университет Иннополис}
\begin{document}

\maketitle

\section{Вводные понятия}

\begin{defn} 
(Случайный процесс) Пусть \((\Omega,\mathscr{F})\) --- 
измеримое
пространство, \(T\) --- любое множество. 
\emph{Вещественно-значным случайным процессом на 
\(T\)} назовем функцию \(\xi=\xi(t,\omega)\) от двух 
перемен \(t\in T\),
\(\omega\in\Omega\) такая, что для каждого фиксированного \(t\), 
функция \(\omega\mapsto \xi(t,\omega)\) из \(\Omega\) в \(\R\) 
является 
случайной величиной.
\end{defn}
Чаще всего переменная \(t\) пишется как нижний индекс:
\[
\xi(t,\omega) = \xi_t(\omega).
\]
Если \(T\) состоит только из одной точки, то случайный
процесс «вырождается» в одну
только случайную величину.
Если \(T\), например, равно множеству \(\{1,2,3\}\) то мы 
имеем дело с трехмерным
случайным вектором.
Если \(T=I\) где \(I\subset \R\) --- интервал,
то мы имеем дело с
процессом \guillemetleft в непрерывном
времени\guillemetright : в большинстве случаев, в самом деле,
множество \(T\) интерпретируется как «время», и тогда 
\(\xi_{t_1}\), \(\xi_{t_2}\), и т.~д.~интерпретируются как
(случайные, ибо они еще зависят от \(\omega\))
значения процесса в моменты времени \(t_1\), \(t_2\) и т.~д.

Конкретные и простейшие примеры случайных процессов
будут приведены ниже. Одна из первых целей \emph{теории}
случайных процессов --- их математически строгое построение.

Стоит напомнить о понятии измеримости функции. 
Если \((\Omega,\mathscr{F})\) и \((Y,\mathcal{Y})\) 
измеримые пространства (множества \(X\) и \(Y\) 
с \(\sigma\)-алгебрами $\mathscr{F}$ и $\mathcal{Y}$ 
соответственно) и \(f:\Omega\to Y\) функция, то она 
называется измеримой когда для всякого \(E\in\mathcal{Y}\),
\begin{equation}\label{eq:uslovieizmerimosti}
\{\omega \in \Omega \ : \ f(\omega) \in E\} \in \mathscr{F}
\end{equation}
для всех \(E\in\mathcal{Y}\).
Случайная величина, напомним, это как раз измеримая 
функция из \(\Omega\) в \(Y=\R\), где \(\mathcal{Y}=
\mathcal{B}(\R)\).

\paragraph{Понятие распределения случайного вектора}
Нам необходимо владеть понятием \emph{образа меры}.
\begin{defn} (Образ меры)
Пусть \((X,\mathcal{X})\) и \((Y,\mathcal{Y})\) 
измеримые пространства (множества \(X\) и \(Y\) 
с \(\sigma\)-алгебрами $\mathcal{X}$ и $\mathcal{Y}$ 
соответственно) и \(f:X\to Y\) измеримая функция, 
а \(\mu\) --- мера на \(\mathcal{Y}\). Тогда 
\emph{образом меры \(\mu\) под функцией \(f\)} 
называется мера на \(\mathcal{X}\), обозначаемая 
через \(\mu_f\), удовлетворяющая:
\begin{align*}
  P_f(E) = \mu(f^{-1}(E)), \qquad E\in \mathcal{Y}.
\end{align*}
Другие обозначения: \(f_{\#}\mu\), \(\mu\circ f^{-1}\).
В случае, когда \(\mu=\mathbb{P}\), обозначают 
и через \(P_{f}\).
\end{defn}
Символ \(f^{-1}\) в этой формуле не обозначает 
обратную функцию от \(f\), которой может не быть. Символ
\(f^{-1}(E)\) обозначает прообраз множества \(E\),
то есть, множество точек \(x\) из \(X\) такие, что 
\(f(x) \in E\).

Можно сказать, что путем функции \(f\) 
мы «воспроизводим» меру \(\mu\), «перенося»
\(\mu\),
определена 
на исходном пространстве, на конечное пространство
таким образом, что новая мера \(\mu_f\) на конечном пространстве
«отражает» меру \(\mu\), принимая такие же значения
на тех измеримых подмножествах множества \(Y\), как и мера 
\(\mu\) на подмножествах множества \(X\),
которые находятся 
в соответствии c ними через функцию \(f\).

\begin{defn} (Распределение случайного вектора)
  Пусть \((\Omega,\mathscr{F},\mathbb{P})\) ---
  вероятностное пространство и \(\xi:\Omega\to\R^d\) 
  (\(d\geq 1\)) --- случайный вектор.
  Его \emph{распределением} называется мера \(P_{\xi}
  =\xi_{\#}\mathbb{P}\) на борелевской \(\sigma\)-алгебре 
  \(\mathcal{B}(\R^d)\). 
\end{defn}
У любого случайного вектора \(\xi\) есть распределение
(как и функция распределения \(F_{\xi}\)).
\begin{example} (Распределение вырожденного случайного вектора
  есть мера Дирака)
  Пусть \(\xi(\omega) = x\) для всех \(\omega\in\Omega\),
  где \(x\in \R^d\) --- фиксированный. Тогда, если 
  \(x\in B\in \mathcal{B}(\R^d)\):
  \begin{align*}
    P_{\xi}(B) = \PP(\xi^{-1}(B)) = 
    \PP(\Omega) = 1,
  \end{align*}
  а если \(x\notin B\in \mathcal{B}(\R^d)\), тогда 
  \begin{align*}
    P_{\xi}(B) = \PP(\emptyset) = 0.
  \end{align*}
  Получается, что \(P_{\xi} = \delta_{x}\), что 
  называется мерой Дирака сосредоточена в
  точке \(x\).
\end{example}

\begin{zadachsem}
Пусть \(\Omega= [0,1]\), \(\mathscr{F}\) = \(\mathcal{B}([0,1])\) 
(борелевская \(\sigma\)-алгебра на интервале \([0,1]\)) и 
\(\PP = \mathscr{L}^1_{[0,1]}\) (лебеговская мера на \(\sigma\)-алгебре 
\(\mathcal{B}([0,1]))\). Возьмем такую случайную величину:
\begin{align*}
  \xi(\omega) = 
  \begin{cases} 
    0 & \quad \textrm{ если } 0\leq \omega <1/3,
    \\
    1 & \quad \textrm{ если } 1/3\leq \omega < 2 /3.
  \end{cases}
\end{align*}
Найти \(P_{\xi}\) (распределение случайной величины \(\xi\)). 
\end{zadachsem}
\begin{remark}\label{remark:funkzraspr}
Из теории вероятности известно, что знание функции распределения 
\(F_{(\xi_1,\xi_2,\ldots,\xi_d)}\) случайного вектора
\((\xi_1,\ldots,\xi_d)\) однозначно определяет его распределение,
и, наоборот, знание распределения \(P_{\xi_1,\ldots,\xi_n}\)
определяет функцию вероятности \(F_{\xi_1,\ldots,\xi_d}\)
(см.~\cite[II-3-Теорема 2]{shiryaev-rus}) 
\end{remark}
\paragraph{Замена переменных в интеграле Лебега}
Важное свойство образа \(f_{\#}\mu\) меры \(\mu\) под измеримой функцией \(f:\Omega\to X\)
 относится к интеграции функции \(g\) из множества \(X\) в \(\R\),
являющейся интегрируемой относительно \(f_{\#}\mu\), потому, что
в таком случае, имеет место формула
\begin{equation}\label{eq:changeofvars}
  \int_{X} g(x) (f_{\#}\mu)(dx) = \int_{\Omega} g\circ f (\omega) \mu(d\omega).
\end{equation}
(см.~\cite[II-6-Теорема 7]{shiryaev-rus}).
Справедливо задаться вопросом --- а как понять, что функция \(g:X\to\R\) интегрируема
по мере \(f_{\#}\mu\), но ответ находится в самом этом только что озвученном 
факте, потому, что если \(g\) --- \(f_{\mu}\)-интегрируема, 
то \eqref{eq:changeofvars} верно, и можно тогда проверить значения интегралов,
скажем, положительной \(g^+\) и отрицательной \(g^-\) частей функции \(g\) 
по \(f_{\#}\mu\) через правую часть уравнения \eqref{eq:changeofvars}; если 
оба значения получаться конечными, тогда это значит,
что \(g\) является \(f_{\#}\mu\)-измеримой.

Из формулы \eqref{eq:changeofvars} следует, например, что
для случайной величины \(\xi:\Omega\to\R\),
\begin{align*}
  \EE \xi = \int_{\R} x P_{\xi}(dx), \qquad 
  \textrm{Var}(\xi) = \int_{\R} (x-\EE\xi)^2 P_{\xi}(dx),
\end{align*}
в случае, конечно, когда \(\EE\xi\) и \(\EE\xi^2\), соответственно,
существуют, что зависит от конечности интегралов 
\(\int_{\Omega}|\xi(\omega)| \PP(d\omega)\) и 
\(\int_{\Omega}\xi^2(\omega)\PP(d\omega)\). Однако, как известно,
если конечно значение второго интеграла, то конечно значение и первого.

\begin{defn} (Сечения и выборочные траектории) 
Пусть \((\xi_t)_{t\in T}\) случайно процесс, задан на \((\Omega,\mathscr{F})\).
Если \(t\in T\) фиксируется, то \emph{сечением} в момент времени \(t\) 
есть случайная величина \(\xi_t\). 
Если \(\omega\in\Omega\) фиксируется, то функция \(t\mapsto \xi_t\)
называется \emph{выборочной траекторией} или просто траекторией. 
\end{defn}
Множество \(\Omega\) содержит, как это принято истолковывать, 
все возможные элементарные события. В этом смысле, исходу \(\omega\in\Omega\)
соответствует целая функция (траектория) \(t\mapsto\xi_t(\omega)\). 
Поэтому случайный процесс можно понимать как случайную функцию из \(T\) 
в \(\Omega\), ведь каждая траектория --- случайная, зависит от \(\omega\).
А можно понимать случайный процесс и как семейство (индексировано точками 
\(t\in T\)) случайных величин. Обозначение процесса через \((\xi_t)_{t\in T}\)
наводит мысль на эту, вторую интерпретацию, нежели интерпретации 
в терминах выборочных 
траекторий.

\begin{example}\label{example:xiequalsomegat}
Рассмотрим процесс $\xi_t(\omega)=t\omega$, \(t\in T= [0,\infty)\),
\(\omega\in \Omega=[0,1]\).
В качестве вероятностного пространства здесь рассматриваем тройку
$(\Omega,\mathscr{F},\mathbb{P})$,
где пространство исходов ${\Omega=[0,1]}$,
$\sigma$-алгебра $\mathscr{F}$ является борелевской:
$\mathscr{F} = \mathcal{B}([0,1])$,
а вероятностная мера совпадает с мерой Лебега на отрезке $[0,1]$,
которую обозначаем через \(\mathscr{L}^1_{[0,1]}\).
Тогда при фиксированном $\omega_0\in[0,1]$
траектория представляет собой линейную функцию
${\xi_t(\omega_0)=\omega_0 t}$, принимающую значение ноль
в момент ${t=0}$ и значение $\omega_0$ в момент $t=1$.
Если же зафиксировать ${t_0\in(0,1]}$,
то получится случайная величина ${\xi_{t_0}(\omega)=t_0\omega}$,
равномерно распределенная на
отрезке $[0,t_0]$, т.е.~${\xi_{t_0}\sim\textrm{Uni}(0,t_0)}$.
При  ${t_0=0}$ получается вырожденная случайная величина:
$\xi_0(\omega)=0$ при всех $\omega\in[0,1]$. 
\end{example}
\begin{figure}[ht]
\centering
\begin{tikzpicture}[scale=1.2, >=Stealth]

    \draw[->] (0,0) -- (5,0) node[right] {$t$};
    \draw[->] (0,0) -- (0,4) node[above] {$\xi_t(\omega)$};
    

    \foreach \w/\pos in {0.2/below left, 0.4/left, 0.6/right, 0.8/right, 1.0/above right} {
        \draw[blue, thick] (0,0) -- (4,4*\w) node[\pos] {$\omega=\w$};
    }
    

    \draw[dashed] (1,0) node[below] {$1$} -- (1,4);
    

    \foreach \w in {0.2,0.4,0.6,0.8,1.0} {
        \fill (1,\w) circle (2pt);
    }
    
    \node[text width=8cm] at (2.5, -1) {
        \footnotesize Траектории процесса $\xi_t(\omega)=t\omega$ для различных $\omega\in[0,1]$.
        При фиксированном $\omega$ траектория --- линейная функция. При фиксированном $t$ 
        сечение --- случайная величина, равномерно распределенная на $[0,t]$.
    };
\end{tikzpicture}
\caption{Выборочные траектории процесса из примера \ref{example:xiequalsomegat}}
\label{fig:trajectories_example1}
\end{figure}
Обращаем внимание читателя на то, что в данном примере траектория
процесса однозначно восстанавливается по ее части: достаточно
узнать наклон траектории на любом интервале времени. На практике же
более типичная ситуация -- это когда предыстория не позволяет однозначно
восстановить траекторию процесса и узнать ее будущее. В таких случаях
различным исходам $\omega_1$ и $\omega_2$ может соответствовать одна и
та же наблюдаемая до текущего момента времени предыстория, но будущее
этих траекторий может отличаться.

\begin{example}\label{example:notpredicted}
  Пусть \(T=\{1,2,3,\ldots\}\),
  \((\Omega,\mathscr{F},\mathbb{P})
  =([0,1],\mathcal{B}(\R),\mathscr{L}^1_{[0,1]})\).
  Определим процесс в явном виде 
  \(\{\xi_j\}_{j=1}^{\infty}\) так:
  \begin{align*}
    \xi_1(\omega) = & \ 
    \begin{cases}
    1 \  & \ \textrm{ если } \omega \in [0,1/2),
    \\
    0 \ & \ \textrm{ если } \omega \in [1/2,1];
    \end{cases}
    \qquad 
    \xi_2(\omega) = 
    \begin{cases}
    1 \  & \ \textrm{ если } \omega \in [0,1/4)\cup(3/4,1],
    \\
    0 \ & \ \textrm{ если } \omega \in [1/4,3/4],
    \end{cases}
    \\
     \xi_3 = & \ \xi_1, \xi_4 = \xi_2, \ldots 
  \end{align*}
Ясно, что каждое сечение, то есть, в этом примере, 
каждая случайная величина \(\xi_j\) имеет распределение 
\(\textrm{Bernoulli}(1/2)\). Допустим, что начинаем наблюдать процесс, 
и выясняется, что 
\begin{align*}
  \xi_1(\omega) = 1.
\end{align*}
Можем ли мы из этого найти \(\omega\), и, таким образом 
знать все будущие значения \(\xi_2,\xi_3,\ldots\)? Нет: 
при, например, \(\omega_1=1/10\) мы получаем \(\xi_1(\omega_1) = 1,
\xi_2(\omega_2) = 1\), но 
при \(\omega_2=3/8\) получаем \(\xi_1(\omega_2) = 1,\xi_2(\omega_2)=0\).
Из информации о том, что \(\xi_1(\omega) = 1\) мы не можем найти 
исход \(\omega\) и, следовательно, не можем узнать всю траекторию.

Заметим, что \(\xi_1\) и \(\xi_2\) можно интерпретировать как результаты 
бросков честной монеты, и нетрудно проверить, что \(\xi_1\) и \(\xi_2\) 
--- независимы. Одна из четырех проверок, которые нужно провести для 
подтверждения независимости такова: 
\begin{align*}
  \PP(\xi_1 = 1, \xi_2=1) = \mathscr{L}^1_{[0,1]}([0,1/2)\cap [0,1/4))
  = \mathscr{L}^1_{[0,1]}([0,1/4)) = \frac{1}{4} = \PP(\xi_1=1)\PP(\xi_2=1). 
\end{align*}
Но, конечно, вся последовательность \(\{\xi_j\}_{j=1}^{\infty}\)
не является последовательностью независимых случайных величин, потому 
что испытания с четными номерами всегда дадут одинаковые исходы,
и аналогично с испытаниями с нечетными номерами. 

Вопрос для более любопытного студента:
можете задать случайный процесс как-то по другому, 
так, чтобы \(\xi_1\), \(\xi_2\) и \(\xi_3\) оказались 
независимыми в совокупности (но все же бернуллиевскими)?
А чтобы также \(\xi_1\), \(\xi_2\), \(\xi_3\) и \(\xi_4\)?
\end{example}
\begin{zadachsem}\label{eq:notpredicted2}
  Либо решая вопрос, задан в конце предыдущего примера, 
  либо другим способом, придумать пример случайного 
  процесса (задав его \emph{явно} как функцию из
  \(T\times \Omega\) в \(\R\)), в котором знание 
  значения \(\xi_1\) и \(\xi_2\) \emph{не} позволяло предсказать 
  значения \(\xi_j\) для \(j=3,4,\ldots\).
\end{zadachsem}

\paragraph{Конечномерные распределения случайного процесса}
Многие случайные процессы, как мы увидим, задаются через
семейство так называемых конечномерных распределений. 
\begin{defn} (Конечномерные распределения случайного процесса)
  Пусть \((\xi_t)_{t\in T}\) --- случайный процесс. 
  Его \emph{конечномерными распределениями} называются распределения
  \(P_{\xi_{t_1},\ldots,\xi_{t_n}}\) случайных векторов
  \(\xi_{t_1},\ldots,\xi_{t_n}\) при любых \(n\in \N\) 
  и всяческих наборах различных точек \(t_1,t_2,\ldots\) из \(T\).
\end{defn}
Напомним, что распределение \(P_{\xi_{t_1},\ldots,\xi_{t_n}}\) 
--- это мера на \((\R^n,\mathcal{B}(\R^n))\), характеризуема 
равенствами  
\begin{align*}
  P_{\xi_{t_1},\ldots,\xi_{t_n}}(A) 
  = \mathbb{P}(\{\omega\in \Omega \ : \ 
  (\xi_{t_1}(\omega),\ldots,\xi_{t_n}(\omega))\in B\})
\end{align*}
для всех \(B\in \mathcal{B}(\R^d)\). Другими словами, 
\(P_{\xi_{t_1},\ldots,\xi_{t_n}} =
(\xi_{t_1},\ldots,\xi_{t_n})_{\#}\mathbb{P}\), образ меры \(\mathbb{P}\) 
под случайным вектором \((\xi_{t_1},\ldots,\xi_{t_n})\).

Наряду с понятием конечномерных распределений случайного процесса,
не менее важны понятия среднего процесса и ковариационной функции процесса.
\begin{defn} (Среднее, ковариационная функция)
  Пусть \((\xi_t)_{t\in T}\) --- случайный процесс, задан 
  на вероятностном пространстве \((\Omega,\mathscr{F},\mathbb{P})\).
  \begin{itemize}
    \item Если для каждого \(t\), существует \(\EE\xi_t\), то 
    функция \(t\mapsto \EE\xi_t\) называется функция среднего 
    процесса, обозначают ее часто через \(m_{\xi}(t)\)
    или \(m(t)\) (если процесс понятен из контекста);
\item если еще существует, для каждой пары \((s,t)\in T\times T\),
ковариация \(\textrm{Cov}(\xi_s,\xi_t)\), то функция 
\(K_{\xi}(s,t) = \textrm{Cov}(\xi_s,\xi_t)\)
из \(T\times T\) в \(\R\) называется 
ковариационной функцией случайного процесса
(\(K(s,t)\) если процесс понятен из контекста). 
  \end{itemize}
\end{defn}
\textit{Предупреждение}. В некоторых текстах 
эту функцию называют \textit{корреляцонной}, 
в то время как ковариационной называют 
функцию \((s,t) \mapsto \EE\xi_s\xi_t\).

\begin{zadachsem}\label{zadachsem:findmeanandcov1}
  Для процессов из примеров \ref{example:xiequalsomegat}
  и \ref{example:notpredicted}, найти \(m(t)\) и \(K(t,s)\).
\end{zadachsem}

\begin{zadachsem}\label{zadachsem:Itomega}
  Пусть \(T=[0,1]\), \(\Omega=[0,1]\),
  \(\mathscr{F}=\mathcal{B}([0,1])\), \(\PP = \mathscr{L}^1_{[0,1]}\), 
  \(\xi_t(\omega) = I(t\leq \omega)\).
  \begin{enumerate}
    \item Найти траектории, сечения и двумерные распределения процесса.
    \item Исследовать сечения на попарную независимость.
    \item Найти функцию среднего и ковариационной функции этого процесса.
    \item Допустим, что мы наблюдаем за этим процессом и видим, 
    что для всех \(t\in [0,t_0]\), \(\xi_t\equiv 1\) (это значит,
    \(\xi_t(\omega) = 1\) для всех \(\omega \in \Omega\)). 
    С какой вероятностью скачок до нуля произойдет на интервале 
    времени \([t_0,t_0+\Delta t]\)? (Разумеется, \(\Delta t < 1-t_0\).)
  \end{enumerate}
\end{zadachsem}
\begin{figure}[ht]
\centering
\begin{tikzpicture}[scale=1.3, >=Stealth]
% Левая панель: Траектории для фиксированных ω
\begin{scope}[xshift=0cm]
    \draw[->] (0,0) -- (4.2,0) node[right] {$t$};
    \draw[->] (0,0) -- (0,3.2) node[above] {$\xi_t(\omega)$};
    \draw[dashed] (0,2) -- (4,2) node[right] {$1$};
    \node[below] at (2,-0.3) {(a) Траектории для фиксированных $\omega$};
    
    % Сетка
    \foreach \x in {0.5,1,1.5,2,2.5,3,3.5,4} {
        \draw[gray!30] (\x,0) -- (\x,3);
    }
    \foreach \y in {0.5,1,1.5,2,2.5,3} {
        \draw[gray!30] (0,\y) -- (4,\y);
    }
    
    % Траектории для разных ω
    % ω=0.2
    \draw[very thick, blue] (0,2) -- (0.8,2) -- (0.8,0) -- (4,0);
    \node[blue, fill=white, inner sep=1pt] at (0.4,2.2) {$\omega=0{,}2$};
    \fill[blue] (0.8,2) circle (2pt);
    \draw[blue, dashed] (0.8,0) node[below] {$0{,}2$} -- (0.8,2);
    
    % ω=0.5
    \draw[very thick, red] (0,2) -- (2,2) -- (2,0) -- (4,0);
    \node[red, fill=white, inner sep=1pt] at (1,2.4) {$\omega=0{,}5$};
    \fill[red] (2,2) circle (2pt);
    \draw[red, dashed] (2,0) node[below] {$0{,}5$} -- (2,2);
    
    % ω=0.8
    \draw[very thick, green!70!black] (0,2) -- (3.2,2) -- (3.2,0) -- (4,0);
    \node[green!70!black, fill=white, inner sep=1pt] at (1.6,2.6) {$\omega=0{,}8$};
    \fill[green!70!black] (3.2,2) circle (2pt);
    \draw[green!70!black, dashed] (3.2,0) node[below] {$0{,}8$} -- (3.2,2);
    
    % Подписи осей
    \node at (-0.3,2) {$1$};
    \node at (-0.3,0) {$0$};
\end{scope}

% Правая панель: Сечение при фиксированном t и его распределение
\begin{scope}[xshift=5cm]
    \draw[->] (0,0) -- (4.2,0) node[right] {$\omega$};
    \draw[->] (0,0) -- (0,3.2) node[above] {$\xi_t(\omega)$};
    \draw[dashed] (0,2) -- (4,2) node[right] {$1$};
    \node[below] at (2,-0.3) {(b) Сечение при фиксированном $t=0{,}6$};
    
    % Сетка
    \foreach \x in {0.5,1,1.5,2,2.5,3,3.5,4} {
        \draw[gray!30] (\x,0) -- (\x,3);
    }
    \foreach \y in {0.5,1,1.5,2,2.5,3} {
        \draw[gray!30] (0,\y) -- (4,\y);
    }
    
    % Вертикальная линия t=0.6
    \draw[very thick, purple, dashed] (2.4,0) node[below] {$0{,}6$} -- (2.4,3);
    
    % Сечение ξ_t(ω) для t=0.6
    \draw[very thick, purple] (0,0) -- (2.4,0) -- (2.4,2) -- (4,2);
    
    % Заполнение областей для наглядности вероятностей
    \fill[purple!20, opacity=0.5] (0,0) rectangle (2.4,2);
    \fill[purple!20, opacity=0.3] (2.4,0) rectangle (4,2);
    
    % Подписи вероятностей
    \node[purple] at (1.2, 0.5) {$\omega < 0{,}6$};
    \node[purple] at (3.2, 0.5) {$\omega \geq 0{,}6$};
    \node[purple] at (1.2, 2.5) {$\xi_{0{,}6}=0$};
    \node[purple] at (3.2, 2.5) {$\xi_{0{,}6}=1$};
    
    % Подписи осей
    \node at (-0.3,2) {$1$};
    \node at (-0.3,0) {$0$};
\end{scope}

% Нижняя панель: Распределение сечения
\begin{scope}[xshift=3cm, yshift=-4.5cm]
    \draw[->] (0,0) -- (4.2,0) node[right] {значение с.~в.~$\xi_{0{,}6}$};
    \draw[->] (0,0) -- (0,3.2) node[above] {Вероятность};
    \node[below] at (2,-0.4) {(c) Распределение с.~в.~$\xi_{0{,}6}$:
    \textrm{Bernoulli}(0{,}4)};
    
    % Сетка
    \foreach \x in {1,2,3,4} {
        \draw[gray!30] (\x,0) -- (\x,3);
    }
    \foreach \y in {0.5,1,1.5,2,2.5,3} {
        \draw[gray!30] (0,\y) -- (4,\y);
    }
    
    % Столбцы распределения
    % Для ξ=0
    \draw[fill=purple!50, opacity=0.7] (0.5,0) rectangle (1.5,{3*0.6});
    \node at (1, -0.3) {$0$};
    \node at (1, {3*0.6+0.2}) {$P(\xi_{0{,}6}=0)=0{,}6$};
    
    % Для ξ=1
    \draw[fill=purple!50, opacity=0.7] (2.5,0) rectangle (3.5,{3*0.4});
    \node at (3, -0.3) {$1$};
    \node at (3, {3*0.4+0.2}) {$P(\xi_{0{,}6}=1)=0.4$};

    
    % Подписи осей
    \foreach \y/\val in {3/1.0} {
        \node[left] at (0,\y) {$\val$};
    }
\end{scope}
\end{tikzpicture}
\caption{рисунок к задаче \ref{zadachsem:Itomega}: траектории и сечения процесса $\xi_t(\omega) = I(t \leq \omega)$}
\label{fig:indicator_process}
\end{figure}

В связи с интерпретацией процесса как случайные траектории, полезно 
ввести понятия \textit{\textbf{случайного элемента}}. Любую измеримую 
функцию \(f: \Omega \to E\) из измеримого пространства \(\Omega\) 
с \(\sigma\)-алгеброй \(\mathscr{F}\) в измеримое пространство 
\(E\) с \(\sigma\)-алгеброй \(\mathcal{E}\) называют 
\textit{случайным элементом} множества \(E\). Идея понятна: 
«случайность» заключается в том, что элемент \(e\in E\) который 
мы видим, зависит от \(\omega\) (если \(e=f(\omega)\) для какого-то 
\(\omega\in \Omega\)). Еще раз, напомним что
измеримость функции при этих обозначениях означает
выполнение условия \eqref{eq:uslovieizmerimosti} для всех 
\(E\in \mathcal{E}\).

Теперь понятно, что для развития интерпретации
случайного процесса
как \textbf{случайную траекторию}, то есть как 
случайный элемент 
множества \(\R^T\) (множество всех функции 
из \(T\) в \(\R\)) нам
не хватает \(\sigma\)-алгебры на этом множестве. 
Для этого вводится понятие \emph{\textbf{элементарного цилиндра}
с основаниями \(B_1,\ldots,B_n\), привязанными к
точкам \(t_1,\ldots,t_n\) из \(T\)},
где \(n\) --- любое натурально число и
\(B_k\) --- борелевское подмножество 
вещественной прямой:
\begin{equation}\label{eq:defcyl}
C_{t_1,\ldots,t_n}(B_1\times \cdots \times B_n)
= 
\{ f \in \R^T \ : \ f(t_1)\in B_1,\ldots,f(t_n)\in B_n\}.
\end{equation}
То есть цилиндр состоит из
функций \(f:T\to\R\), которые
«в моменты времени \(t_1,\ldots,t_n\)» должны 
проходит «через окна» \(B_1,\ldots,B_n\) соответственно,
а при остальных \(t\) они могут принимать любые значения. 
А \textbf{\textit{цилиндром}} (без прилагательного 
\textit{элементарный}) называют подмножество 
\begin{align*}
C_{t_1,\ldots,t_n}(B)
= 
\{ f \in \R^T \ : \ (f(t_1),\ldots,f(t_n))\in B\},
\end{align*}
где \(B\in\mathcal{B}(\R^n)\).

Семейство всех элементарных цилиндров не является 
алгеброй, но семейство всех цилиндров --- да. Тем 
не менее, эти семейства порождают одну и ту же 
\(\sigma\)-алгебру.
\begin{defn}\label{defn:borelsigmaalgebraonRT}
Борелевской \(\sigma\)-алгеброй \(\mathcal{B}(\R^T)\)
на множестве \(\R^T\) называется наименьшая
\(\sigma\)-алгебра, содержащая все цилиндры.
\end{defn}
\begin{figure}[ht]
\centering
\begin{tikzpicture}[scale=1.2, >=Stealth]
    \draw[->] (0,0) -- (6,0) node[right] {$t$};
    \draw[->] (0,0) -- (0,4) node[above] {$x$};
    
    \coordinate (t1) at (2,0);
    \coordinate (t2) at (4,0);
    
    \draw[dashed] (t1) -- (2,4) node[above] {$t_1$};
    \draw[dashed] (t2) -- (4,4) node[above] {$t_2$};
    \fill[blue!20, opacity=0.7] (1.8,1.5) rectangle (2.2,2.5);
    \fill[red!20, opacity=0.7] (3.8,1.0) rectangle (4.2,2.0);
    
    \node at (2,2.7) {$B_1$};
    \node at (4,2.3) {$B_2$};
    
    \draw[thick, green!50!black] (0,0.5) to[out=20, in=180] (2,2) to[out=0, in=160] (4,1.5) to[out=20, in=180] (5.5,3);
    \draw[thick, orange] (0,1.2) to[out=10, in=190] (2,1.8) to[out=10, in=170] (4,1.2) to[out=10, in=180] (5.5,2.5);
    \draw[thick, purple] (0,2.0) to[out=0, in=200] (2,2.2) to[out=20, in=150] (4,1.8) to[out=30, in=190] (5.5,3.2);
    
    \draw[dashed, blue] (1.8,1.5) -- (1.8,2.5);
    \draw[dashed, blue] (2.2,1.5) -- (2.2,2.5);
    \draw[dashed, red] (3.8,1.0) -- (3.8,2.0);
    \draw[dashed, red] (4.2,1.0) -- (4.2,2.0);
    
    \node[text width=8cm] at (3, -1) {
        \footnotesize Цилиндр $C_{t_1,t_2}(B_1\times B_2)$ состоит из всех функций $f:T\to\mathbb{R}$, 
        таких что $f(t_1)\in B_1$ и $f(t_2)\in B_2$. На рисунке показаны три траектории, 
        принадлежащие этому цилиндру (проходят через «окна» $B_1$ и $B_2$ в моменты $t_1$ и $t_2$).
    };
\end{tikzpicture}
\caption{Иллюстрация цилиндра в пространстве траекторий}
\label{fig:cylinder_example}
\end{figure}

Если \(T={0,1,2,\ldots}\) или 
\(T=\{1,2,\ldots\}\) обозначают \(\R^T\) и 
\(\mathcal{B}(\R^T)\) через 
\(\R^{\infty}\) и \(\mathcal{B}(\R^{\infty})\) 
соответственно или же с символом \(\N\) вместо 
\(\infty\). В таком случае, напомним, 
элемент из \(\R^{\infty}\) понимается 
как последовательность вещественных чисел. 
\begin{zadachsem}\label{zadachesem:limitsareinborelsigmaalg}
  Доказать, что следующие множества принадлежат \(\sigma\)-алгебре 
  \(\mathcal{B}(\R^{\infty})\):
  \begin{enumerate}
    \item \(\{x\in\R^{\infty} \ : \ \sup_{x_n} > a \}\),
    \item \(\{x\in \R^{\infty} \ : \ \limsup x_n \leq a\}\),
    \item \(\{x\in \R^{\infty} \ : \ \lim_{n\to\infty} x_n 
    \text{ существует и он конечен }\}\),
    \item \(\{x\in \R^{\infty} \ : \ \sum_{n=1}^{\infty}|x_n| > a\}\).
  \end{enumerate}
  \textit{Подсказка}. Объединение множеств относится 
  к оператору «существует», а пересечение --- к оператору 
  «для всех».
\end{zadachsem}
Как было упомянуто выше, многие процессы 
задаются через задание их конечномерных
распределений. Их существование обеспечивается 
теоремой Колмогорова (Теорема \ref{thm:kolmogorovsogext}
ниже), в которой процесс, с желаемыми конечномерными 
распределениями строится 
на пространстве \((\Omega,\mathscr{F})\) где 
\(\Omega  = \R^T\) и \(\mathscr{F}=\mathcal{B}(\R^T)\).
Но можно сразу сказать, что такая конструкция сама 
по себе не
достаточна если хочется обосновать существование
процессов, которые
владели бы такими свойствами, как непрерывность,
или монотонность, или неотрицательность, и так далее
--- свойства, широко возникающие в практике. Причиной 
является следующая теорема \cite[II-2-Теорема 5]{shiryaev-rus}:
\begin{thm}\label{thm:shiryaevII-2-5}
Любой элемент \(\sigma\)-алгебры 
определяется \guillemetleft ограничениями\guillemetright,
наложенными на функции \((x_t)_{t\in T}\) не более чем
в счетном числе точек из множества \(T\), то есть
\begin{align*}
  A \in \mathcal{B}(\R^T) 
  \Longleftrightarrow \exists \{t_j\}_{j=1}^{\infty} 
  \subset T 
  \text{ и }  B\in \mathcal{B}(\R^{\infty})
  \text{ такие, что } A = 
  \{x\in \R^T \ : \ (x_{t_1},x_{t_2},\ldots) \in B\}.
\end{align*}
\end{thm} 
Для понимания недостаточности, упомянутой выше,
полезно разобрать следующую задачу, используя эту теорему.
\begin{zadachsem}\label{zadachasem:notinsigmaalgebra1}
Какие из следующих множеств принадлежит 
\(\sigma\)-алгебре \(\mathcal{B}(\R^{[0,1]})\)?
\begin{enumerate}
  \item \(\{x\in \R^{[0,1]} \ : \ 
  x_t = 0 \ \forall t\in [0,1]\}\),
  \item \(\{x\in \R^{[0,1]} \ : \ 
  x_t = > c \ \forall t\in [0,1]\}\),
  \item \(\{x\in \R^{[0,1]} \ : \ 
  x_t = 0  \ \text{ хотя бы для одного } t\in [0,1]\}\),
  \item \(\{x\in \R^{[0,1]} \ : \ 
  x \text{ непрерывна в фиксированной точке } t_0 \in [0,1]\}\).
\end{enumerate}
\end{zadachsem}
Перед обсуждением теорем Колмогорова 
(Теорема \ref{thm:kolmogorovsogext} и 
Теорема \ref{thm:kolmogorovcontver} ниже)
нам понадобятся 
два весьма естественных понятия: 1) 
\textit{распределение случайного процесса}
и 2) \textit{процесс координатных проекций}.

\paragraph{Распределение случайного процесса}
Пусть \((\Omega,\mathcal{F},\mathbb{P})\) 
--- вероятностное пространство и 
\((\xi_t)_{t\in T}\) случайный процесс на нем. 
Множество функций \(\R^T\) возьмем 
с \(\sigma\)-алгеброй \(\mathcal{B}(\R^T)\).
\textit{Распределением случайного процесса}
\((\xi_t)_{t\in T}\) называется 
образ \(\operatorname{trj}_{\#}\PP\) меры \(\PP\) под 
функцией \(\operatorname{trj}\)
и обозначается \(P_{\xi}\), 
где \(\operatorname{trj}\)
определена так:
\begin{align*}
  \operatorname{trj} : \ \Omega & \longrightarrow 
  \R^{T} 
  \\
  \omega & \longmapsto 
  \left\{
\begin{aligned}
\operatorname{trj}(\omega): \ T & \longrightarrow \R 
\\ 
 t & \longmapsto \xi_t(\omega).
\end{aligned}
  \right.
\end{align*}
То есть, словами, \(\operatorname{trj}\) это просто функция, 
которая каждому элементарному событию 
\(\omega\) сопоставляет соответствующую выборочную 
траекторию \(t\mapsto \xi_t(\omega)\). 
Разумеется, нужно сначала проверить, что
\(\operatorname{trj}\) --- измеримая функция,
иначе, понятие образа меры под этой функцией бессмысленно.
Но это легко: если \(C=C_{t_1,\ldots,t_n}(B_1\times\cdots
\times B_n)\) --- цилиндр, то 
\begin{align*}
\operatorname{trj}^{-1}(C) = & \ 
\{\omega \in \Omega \ : \ 
\operatorname{trj}(\omega) \in C\} 
=  \{\omega \in \Omega \ : \ 
\xi_{t_1}\in B_1, \ldots, \xi_{t_n}\in B_n
\} 
\\
= & \ \xi_{t_1}^{-1}(B_1)\cap \ldots \cap 
\xi_{t_n}^{-1}(B_n) \in \mathcal{F},
\end{align*}
ибо каждая \(\xi_{t_j}\) --- случайная величина. 
Получается, что для каждого подмножества 
\(\R^T\), из класса подмножеств, порождающего 
\(\sigma\)-алгебру \(\mathcal{B}(\R^T)\),
проверяется условие измеримости 
\eqref{eq:uslovieizmerimosti}. Из следующего упражнения,
следует и справедливость условия измеримости 
и для всех элементов \(\sigma\)-алгебры \(\mathcal{B}(\R^T)\).
\begin{zadachsem}\label{zadachsem:izmerimostnageneratorakh}
 Пусть \((\Omega,\mathscr{F})\) и \((Y,\mathcal{Y})\) 
измеримые пространства (множества \(X\) и \(Y\) 
с \(\sigma\)-алгебрами $\mathscr{F}$ и $\mathcal{Y}$ 
соответственно), причем \(\sigma\)-алгебра
\(\mathcal{Y}\) на \(Y\) порождается каким-то
подсемейством \(\mathcal{Y}_0\) подмножеств из \(Y\), 
что обозначается как \(\sigma(\mathcal{Y}_0)=\mathcal{Y}\).
Пусть \(f:\Omega\to Y\) функция. Если 
\begin{equation*}
\{\omega \in \Omega \ : \ f(\omega) \in E\} \in \mathscr{F}
\end{equation*}
для всех \(E\in\mathcal{Y}_0\), то это верно и для 
всех \(E\in \mathcal{Y}\).
\end{zadachsem}
Значит \(\operatorname{trj}\) измерима и
для всякого \(E\in \mathcal{B}(\R^T)\), 
\begin{align*}
  P_{\xi}(E) = 
  \PP(\{\omega \in \Omega \ : \ 
  \text{ функция } t\mapsto \xi_t(\omega) 
  \text{ принадлежит к } E\}).
\end{align*}

\paragraph{Процесс координатных проекций}
Допустим есть теперь только \(\R^T\) с \(\sigma\)-алгеброй 
\(\mathcal{B}(\R^T)\). Если \(\omega\in \R^T\) --- любой элемент,
то есть, \(\omega\) является какой-то \textit{функцией}
из \(T\) в \(\R\), то и для любого \(t\in T\), 
\(\omega(t) \in \R\). Обозначим теперь \(\R^T\) через \(\Omega\) 
и \(\mathcal{B}(\R^T)\) через \(\mathscr{F}\). Только что сделанное наблюдение 
позволяет определить процесс \((\xi_t)_{t\in T}\) на этом измеримом пространстве 
\((\Omega,\mathscr{F})\) таким образом:
\begin{align*}
  \xi_t(\omega) = \omega(t), \qquad t\in T, \ \omega \in \Omega = \R^T.
\end{align*}
Легко проверить, что каждая функция \(\xi_t\) является случайной 
величиной (упражнение).

\begin{figure}[ht]
\centering
\begin{tikzpicture}[scale=1.2, >=Stealth]
    \draw[->] (0,0) -- (6,0) node[right] {$t$};
    \draw[->] (0,0) -- (0,4) node[above] {$x$};
    
    \node[draw, blue, shape=ellipse, minimum width=3cm, minimum height=2cm] (Omega) at (1.5,2) {$\Omega = \mathbb{R}^T$};
    \node[blue] at (1.5,3.5) {Пространство всех функций};
    
    \draw[thick, red] (0.5,1.0) to[out=20, in=180] (2,2.5) to[out=0, in=180] (3.5,1.5) 
                      to[out=0, in=180] (4.5,2.8) to[out=0, in=160] (5.5,2.0);
    \node[red] at (3,0.7) {$\omega \in \Omega$ (траектория)};
    

    \coordinate (t0) at (3.2,0);
    \draw[dashed] (t0) -- (3.2,4);
    \node at (3.2,-0.3) {$t_0$};
    

    \fill[red] (3.2,1.82) circle (3pt);
    \draw[dashed, red] (3.2,1.82) -- (5,1.82);
    

    \draw[->] (5,1) -- (7,1) node[right] {$\mathbb{R}$};
    \fill[red] (6,1) circle (3pt) node[above] {$\xi_{t_0}(\omega)=\omega(t_0)$};
    

    \draw[->, thick, blue] (Omega) to[bend left=20] node[above] {$\xi_{t_0}$} (6,1);
    
    \node[text width=8cm] at (3.5, -1) {
        \footnotesize Процесс координатных проекций: для каждого $t\in T$ отображение 
        $\xi_t:\Omega\to\mathbb{R}$ сопоставляет траектории $\omega$ её значение
        в момент $t$: $\xi_t(\omega)=\omega(t)$.
    };
\end{tikzpicture}
\caption{Схематическое изображение процесса координатных проекций}
\label{fig:coordinate_projection}
\end{figure}

Значит, одно существование \(\R^T\) с \(\sigma\)-алгеброй 
уже зарождает случайный процесс, однако, конечно, интересен 
только вопрос об его распределении, а без какой-либо меры 
на \((\R^T,\mathcal{B}(\R^T))\) такой вопрос не стоит.
Тогда, пусть \(P_0\) --- произвольная
вероятностная мера 
на \((\R^T,\mathcal{B}(\R^T))\), и пусть
\((\xi_t)_{t\in T}\) --- процесс координатных 
проекций. Тогда распределение этого процесса 
совпадает с \(P_0\), что остается как
очень простое упражнение:
\begin{zadachsem}\label{zadachsem:P0equalsPxi}
  Пусть \(P_0\) --- вероятностная мера 
  на пространстве \((\Omega,\mathscr{F})\),
  где \(\Omega=\R^T\) и 
  \(\mathscr{F}=\mathcal{B}(\R^T)\).
  Пусть \((\xi_t)_{t\in T}\) --- 
  соответствующий процесс координатных 
  проекции. Доказать, что \(P_{\xi}=P_0\).
\end{zadachsem}

Настает пора дать теорему Колмогорова
о существовании процесса с заданными 
конечномерными распределениями. 
Это теорема называется \emph{теоремой Колмогорова
о согласованных распределениях}. Для начала, 
дадим определение --- что мы имеем ввиду 
под «согласованными распределениями».

\begin{center}\emph{Впредь, если не оговариваем об обратном,
  мы будем считать что \(T\) --- 
подмножество полупрямой \([0,\infty)\).}
\end{center}

\begin{defn}\label{defn:consistenfamilydist}
Пусть имеется, для каждого \(n\in \N\) и
каждого набора различных неотрицательных 
чисел \(t_1,\ldots,t_n\) из \(T\),
вероятностная мера \(Q_{t_1,\ldots,t_n}\) 
на \((\R^n,\mathcal{B}(\R^n))\).
Всё семейство этих мер называется 
\emph{семейством конечномерных распределений},
и это семейство называется 
\emph{согласованным}, если выполняются 
эти два условия:
\begin{enumerate}[(i)]
\item для каждого \(n\), каждого 
набора \((t_1,\ldots,t_n)\), любой 
перестановки \((t_{j_1},\ldots,t_{j_n})\) этих 
чисел, и любых \(B_1,\ldots,B_n\in \mathcal{B}(\R)\),
верно:
\begin{align*}
  Q_{t_1,\ldots,t_n}(B_1\times \cdots 
  \times B_n) =
  Q_{t_{j_1},\ldots,t_{j_n}}
  (B_{j_1}\times \cdots 
  \times B_{j_n});
\end{align*}
\item для каждого \(n\geq 2\), каждого 
набора \((t_1,\ldots,t_n)\), любого 
\(B\in \mathcal{B}(\R^{n-1})\),
верно:
\begin{align*}
  Q_{t_1,\ldots,t_n}(B\times \R) 
  = Q_{t_1,\ldots,t_{n-1}}(B).
\end{align*} 
\end{enumerate}
\end{defn}
\begin{zadachkont}\label{zadachcont:fddareconsistent}
Пусть \((\xi_t)_{t\in T}\) --- случайный процесс,
где \(T\subset [0,\infty)\). 
Доказать, что семейство его конечномерных 
распределений образует семейство
согласованных конечномерных распределений. 
\end{zadachkont}
Свойство согласованности для конечномерных 
распределений уже заданного
случайного процесса настолько естественно, 
что сам факт определения понятия согласованности 
может вызывать недоумение. Однако 
нетрудно придумать пример, в котором семейство 
\(\mathcal{Q}\) нарушает условия согласованности.
Если, скажем, \(1,2\in T\) и \(\mathcal{Q}\) 
содержит меры \(P_{1,2}\) и \(P_{2,1}\) такие:
\(P_{1,2}=\mathscr{L}^2_{[0,1]^2}\) (мера 
Лебега ограничена на квадрате, то есть, 
\(\mathscr{L}^2_{[0,1]^2}(B) =
\mathscr{L}^2(B\cap [0,1]^2)\)), а 
\(P_{2,1}\) --- гауссовская мера\footnote{Это мера, задана 
плотностью гауссовского вектора. Например,
\(P_{2,1}(B) = \frac{1}{2\pi}\int_B 
e^{-\frac{1}{2}(x_1^2 + x_2^2)}dx_1dx_2\)
для всех \(B\in \mathcal{B}(\R^2)\).} на
 \((\R^2,\mathcal{B}(\R^2))\), то, очевидно,
 \(\mathcal{Q}\) уже нарушает первое условие 
 согласованности.



\begin{thm}\label{thm:kolmogorovsogext}
  (Колмогорова о согласованных 
  распределениях)
Пусть \(T\subset [0,\infty)\) и
\(\mathcal{Q}\) --- семейство 
согласованных конечномерных распределений. 
Тогда существует \((\Omega,\mathscr{F},\mathbb{P})\)
и процесс \((\xi_t)_{t\in T}\) на нем, у 
которого семейство конечномерных
распределений равно семейству \(\mathcal{Q}\).
\end{thm}
В самом деле нет никакой тайны по поводу 
того, какими оказываются 
\(\Omega\) и \(\mathscr{F}\) в доказательстве 
этой теоремы: \(\R^T\) и \(\mathcal{B}(\R^T)\).
Легко проверить, что определяя 
функцию \(P_0\) на цилиндрах ровно так, 
как это «прописано» соответствующей 
мерой из семейства \(\mathcal{Q}\)
(то есть, 
цилиндру \(C=C_{t_1,\ldots,t_n}(B_1\times\cdots \times B_n)\)
дать значение \(P_0(C) := P_{t_1,\ldots,t_n}(B_1\times B_n)\)),
\(P_0\) оказывается счетно-аддитивной функцией 
на совокупности всех конечных объединений
элементарных цилиндров (что еще не является \(\sigma\)-алгеброй),
и далее
доказательство занимается конечно-аддитивным свойством
теперь на алгебре всех цилиндров, за которым следует 
(применением теоремы Каратеодори) расширение 
этой конечно-аддитивной функции \(P_0\)  
до (вероятностной) меры на всю 
\(\sigma\)-алгебру \(\mathcal{B}(\R^T)\).
А поскольку в доказательстве теоремы
роль множество элементарных событий 
\(\Omega\) играет множество функций 
\(\R^T\), то и сам процесс \((\xi_t)_{t\in T}\) 
определяется как процесс координатных проекции, 
дабы воспользоваться 
совпадением (задача \ref{zadachsem:P0equalsPxi})
построенной мерой \(P_0\) и 
распределением \(P_{\xi}\) процесса, определенного 
таким образом. 




Эта теорема, хоть и фундаментальная, 
еще не решает проблему --- 
как построить процесс, с заданными 
конечномерными распределениями, но с каким-то 
еще желаемым свойством, например, 
монотонность, или непрерывность. Если,
например, множество 
\(C([0,1])\) всех непрерывных 
функций было бы \(\mathcal{B}(\R^{[0,1]})\)-измеримым,
то можно было бы: 1) получить, теоремой 
Колмогорова, процесс \((\xi_t)_{t\in T}\) 
с требованными конечномерными распределениями,
2) распределение \(P_{\xi}\) этого процесса 
ограничить на множестве \(C([0,1])\)
(возможно, нормировав меру),
затем забыть обо \(\R^T\setminus C([0,1])\) 
и радоваться новому вероятностному
пространству непрерывных функции, у 
которого процесс координатных 
проекции состоял бы уже целиком из 
непрерывных траекторий. Но, увы,
\(C[0,1]\), и другие полезные для приложений 
подмножества --- не измеримы относительно 
\(\sigma\)-алгебры \(\mathcal{B}(\R^{[0,1]})\).
Поэтому теоретикам пришлось еще поработать, чтобы добиться 
оправданного обсуждения таких объектов.

В связи с этим необходимо ввести понятие
\textit{модификацией} (также называется \textit{версией}),
случайного процесса. 
\begin{defn}\label{defn:modificationofprocess}
  (Стохастическая эквивалентность и неразличимость)
  Пусть \((\xi_t)_{t\in T}\) --- случайный процесс на пространстве 
  \((\Omega,\mathscr{F},\PP)\). 
  \begin{enumerate} 
  \item Случайный процесс \((\eta)_{t\in T}\),
  определен на
  этом же пространстве, называется \emph{версией} или \emph{модификацией}
  случайного процесса \((\xi_t)_{t\in T}\) если для каждого 
  \(t\in T\), множество 
  \[
  N_t = \{\omega \in \Omega \ | \ 
  \xi_t(\omega)\neq \eta_t(\omega) \}
  \]
  \(\mathscr{F}\)-измеримо и \(\PP(N_t) = 0\).
  В таком случае еще говорят, что 
  \((\eta)_{t\in T}\) и \((\xi_t)_{t\in T}\) --- 
  \emph{стохастически эквивалентные}.
  \item Если для случайного процесса \((\eta_t)_{t\in T}\), 
  существуем \(N\in\mathscr{F}\), такое что \(\PP(N)=0\) и
  для всех \(\omega\in \Omega\setminus N\), 
  функции \(t\mapsto \xi_t(\omega)\) и \(t\mapsto \eta_t(\omega)\)
  одинаковы, то процессы \((\xi_t)_{t\in T}\) и \((\eta_t)_{t\in T}\)
  называются \emph{неразличимыми}.
\end{enumerate}
\end{defn}
Кратко, говорят что \((\eta_t)_{t\in T}\) есть модификация 
процесса \((\xi_t)_{t\in T}\) если 
для всех \(t\), \(\xi_t = \eta_t\) п.~н.~---«почти наверное», 
то есть везде кроме, возможно, на измеримом множества меры нуль. Но 
в определении этого понятия это множество может меняться для разных \(t\).

А эти два процесса неразличимы если п.~н.~все их траектории идентичные. 

Следующий факты доказываются очень легко:
\begin{enumerate}
  \item если два процесса неразличимы, то они стохастически
  эквивалентные;
  \item у стохастически эквивалентных процессов одинаковые 
  конечномерные распределения;
  \item стохастическая эквивалентность --- отношение эквивалентности
  на множестве всех случайных процессов на \((\Omega,\mathscr{F},\mathbb{P})\).
  Также неразличимость.
\end{enumerate}
Но, вообще, стохастическая эквивалентность не влечет неразличимость:
\begin{example}\label{example:indicatortequalomega}
  Пусть \((\Omega,\mathscr{F},\PP) = ([0,1],\mathcal{B}([0,1]),
  \mathscr{L}^1_{[0,1]})\),
  \(\xi_t(\omega) := 0\) для всех \(t\) и \(\omega\),
  \(\eta_t(\omega) := I(\omega=t)\). Эти процессы 
  стохастически эквивалентные:
  \begin{align*}
    \text{ для } t\in T, \quad 
    \{\omega \in \Omega \ : \ \xi_t\neq \eta_t(\omega)\} 
    = \{t\},
  \end{align*}
  а \(\mathbb{P}(\{t\}) = 0\).
  Но для каждого \(\omega\in \Omega\), есть одна точка (а именно, 
  \(t=\omega\)) где траектории не совпадают (одна принимает значения 
  \(0\) там, другая \(1\)). Эти процессы не только неразличимы, 
  они даже «везде различимы». 
  
  А еще, у \((\xi_t)_{t\in T}\) все траектории непрерывные,
  но у \((\eta_t)_{t\in T}\) все траектории разрывные.
\end{example}
Однако бывает, что если рассматривать эти понятия в более узком классе
процессов, они совпадают:
\begin{zadachkont}\label{zadachkont:fromrightthenthesame}
  Если у процессов \((\xi_t)_{t\in T}\) и \((\eta_t)_{t\in T}\),
  определенных на одном и том же вероятностном пространстве, 
  п.~н.~траектории непрерывные справа, и эти процессы стохастически
  эквивалентные, то они неразличимы.
\end{zadachkont}
Приведем теперь теорему Колмогорова и сделаем 
несколько комментариев касательно нее.
\begin{thm}\label{thm:kolmogorovcontver} (Колмогорова о непрерывной 
  версии процесса)
Пусть \(T=[0,\infty)\), \((\xi_t)_{t\in T}\) --- случайный процесс,
задан на вероятностном пространстве \((\Omega,\mathscr{F},\PP)\). Если
существуют числа \(\alpha,\beta,C>0\) при которых, для 
всех \(s,t\in T\), 
\begin{align*}
  \EE|\xi_s-\xi_t|^{\alpha} \leq C |t-s|^{1+\beta},
\end{align*}
то существует непрерывная модификация \((\eta)_{t\in T}\) 
этого процесса, для которого 
\begin{align*}
  \PP\big( \big\{ \omega \in \Omega \ : \
  \sup_{\substack{ 0<t-s<h(\omega)\\ s,t\in [0,\infty) }}
  \frac{|\eta_t(\omega)-\eta_s(\omega)|}{|t-s|^{\gamma}} 
  \leq \delta
  \big\}  \big) = 1
\end{align*}
для любого \(\gamma\in (0,\frac{\beta}{\alpha})\) и 
для какой-то константы \(\delta\) и функции \(h\),
положительна для \(\mathbb{P}\)-п.в.~\(\omega\).
\end{thm}



\printbibliography

\end{document}
